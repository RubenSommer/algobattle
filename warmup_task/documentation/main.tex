\documentclass[11pt]{article}

\usepackage{sectsty}
\usepackage{graphicx}

% Margins
\topmargin=-0.45in
\evensidemargin=0in
\oddsidemargin=0in
\textwidth=6.5in
\textheight=9.0in
\headsep=0.25in

\title{Warmup task - Pairsum}
\author{ Ruben }
\date{\today}

\begin{document}
\maketitle	

\section*{Solver}
\subsection*{Go through all pairs}
\begin{itemize}
    \item Go through all pairs in the list
    \item Worst case: $O(n^4)$
\end{itemize}
\subsection*{Hashing (what we used)}
\begin{itemize}
    \item We iterate over all pairs $(i,j)$ and for every pair we calculate the sum $s = L[i]+L[j]$, which is saved in a hashmap.
    \item If we later find a pair $(i',j')$, which add up to the same sum, we check whether the indices are disjunct. If this is the case, the indices are our solution.
    \item Worst case: $O(n^2)$
\end{itemize}
\subsection*{Divide and Conquer}
\begin{itemize}
    \item 
\end{itemize}
\section*{Generator}
\begin{itemize}
    \item Generate $max\_size$ random numbers and put them in a list
    \item Create a pairsum pair
    \item Create ''fake pairs'': pairs which are near to being a pairsum instance and roughly the same
    \item Randomize the list
\end{itemize}
\end{document}